\documentclass[11pt,a4paper]{article}
\usepackage[margin=1in]{geometry}
\usepackage{amsmath,amssymb,amsthm}
\usepackage{pgfplots}
\usepackage{hyperref}
\usepackage{graphicx}
\usepackage{authblk}
\usepackage{booktabs}
\usepackage{caption}
\usepackage{float}
\usepackage{siunitx}
\usepackage{filecontents}
\usepackage{csvsimple}
\usepackage{enumitem}

\pgfplotsset{compat=1.18}

% Veri Dosyaları
\begin{filecontents*}{desi_fs8.csv}
z,fs8,err
0.65,0.462,0.036
0.80,0.436,0.037
0.95,0.410,0.038
\end{filecontents*}

\begin{filecontents*}{grch_fs8.csv}
z,fs8
0.65,0.448
0.80,0.445
0.95,0.435
\end{filecontents*}

\begin{filecontents*}{cmb_grch.csv}
ell,Dl
2,6000
30,5700
100,5600
500,3000
1000,1000
2000,200
\end{filecontents*}

\begin{filecontents*}{cmb_lcdm.csv}
ell,Dl
2,5990
30,5690
100,5590
500,2990
1000,990
2000,195
\end{filecontents*}

\begin{filecontents*}{planck_tt.csv}
ell,tt,tt_err
30,5700,50
100,5600,40
500,3000,30
1000,1000,20
2000,200,15
\end{filecontents*}

\title{The Geometric Residual Curvature Hypothesis (GRCH): \\
A Non-Dark-Matter Explanation of Cosmic Structure Growth}

\author{Mehmet Aslan}
\affil{Independent Researcher, London, United Kingdom}
\date{November 8, 2025}

\begin{document}

\maketitle

\begin{abstract}
We propose the Geometric Residual Curvature Hypothesis (GRCH), a theoretical framework where dark-matter-like effects arise from persistent, large-scale curvature remnants of early cosmic dynamics rather than unseen matter. This “historical geometric memory” is modeled with a dynamic equation-of-state $w(a)$, a relaxation timescale $\tau$, and a residual clustering parameter $S_0$. The model reproduces late-time structure growth, lensing, and $E_G$ measurements from DESI Year 1 and KiDS-Legacy, while remaining consistent with the Planck CMB power spectrum. We derive the phenomenological law from a field-theoretic action containing $R^2$ and $R\square^{-1}R$ terms, derive the memory-relaxation dynamics, and interpret $\rho_{\rm mem}$ as a form of geometric hysteresis or curvature thermodynamics arising from incomplete decoherence of early-universe curvature modes.
\end{abstract}

\section{Introduction}
The cosmological concordance model ($\Lambda$CDM) accurately describes large-scale observations but relies on an undetected dark-matter component comprising roughly 85\% of the total matter density. Despite intensive searches, no dark-matter particle has been found. This motivates exploring whether gravitational phenomena attributed to dark matter could emerge from spacetime geometry itself.

Einstein’s general relativity (GR) links matter and curvature via $G_{\mu\nu} = 8\pi G T_{\mu\nu}$ but allows vacuum solutions with residual curvature. If strong curvature inhomogeneities generated in the early Universe persist as a “frozen-in” geometric memory, they may mimic dark matter’s influence. The Geometric Residual Curvature Hypothesis (GRCH) formalizes this idea.

\section{Phenomenological Framework}
We adopt a spatially flat FLRW background containing an additional curvature-memory energy density $\rho_{\rm mem}(a)$ that obeys
\begin{equation}
\frac{d\rho_{\rm mem}}{d \ln a} = -3(1 + w(a))\rho_{\rm mem} - \frac{\rho_{\rm mem}}{H\tau},
\end{equation}
where $w(a)$ is a dynamic equation-of-state and $\tau$ a relaxation timescale describing geometric memory decay.

We parametrize
\begin{equation}
w(a) = -\frac{1}{2} \left[1 + \tanh\left(\frac{\ln a - \ln a_t}{\Delta}\right)\right],
\end{equation}
with $a_t$ the transition epoch and $\Delta$ its width. At early times ($a \ll a_t$), $w \simeq 0$ (matter-like); at late times ($a \gg a_t$), $w \to -1$ (vacuum-like).

The Friedmann equation reads
\begin{equation}
H^2(a) = H_0^2 \left[ \Omega_b a^{-3} + \Omega_r a^{-4} + \Omega_\Lambda + \frac{\rho_{\rm mem}(a)}{\rho_{\rm crit,0}} \right].
\end{equation}

\subsection{Residual Clustering Factor}
Residual curvature does not cluster as efficiently as cold matter. We introduce a scale-dependent clustering factor $S(a)$:
\begin{equation}
S(a) = S_0 + (1 - S_0) \frac{1 - \tanh[(\ln a - \ln a_t)/0.5]}{2},
\end{equation}
transitioning from $S \simeq 1$ at early times to $S_0 < 1$ today.

The linear growth rate $f = d \ln D / d \ln a$ satisfies
\begin{equation}
\frac{df}{d \ln a} + f^2 + \left[2 + \frac{d \ln H}{d \ln a}\right] f = \frac{3}{2} \Omega_{\rm cl}(a),
\end{equation}
where
\begin{equation}
\Omega_{\rm cl}(a) = \Omega_b a^{-3} + \frac{\rho_{\rm mem}(a) S(a)/\rho_{\rm crit,0}}{H^2(a)/H_0^2}.
\end{equation}

\section{Field-Theoretic Derivation}
To ground Eq. (1) in a covariant theory, we extend GR by adding a curvature-memory term:
\begin{equation}
S = \int d^4x \sqrt{-g} \left[ \frac{M_{\rm Pl}^2}{2} R + \frac{\alpha}{2} R^2 + \frac{m^2}{6} R \square^{-1} R \right] + S_b.
\end{equation}

The nonlocal $R \square^{-1} R$ term introduces a memory kernel with mass scale $m$, localized by an auxiliary field $U$ satisfying
$\square U = R$. Variation of the action in FLRW background leads to
\begin{equation}
\dot{\rho}_{\rm mem} + 3H(1 + w)\rho_{\rm mem} = -\frac{\rho_{\rm mem}}{\tau_{\rm eff}},
\quad \tau_{\rm eff}^{-1} \simeq \frac{m^2}{3H}.
\end{equation}
thereby producing Eq. (1) dynamically. For $m \sim H_0$ the effective relaxation time is $\tau_{\rm eff} \sim H^{-1}_0$, and $\alpha > 0$ ensures ghost freedom.

\subsection{Quantum-Memory Interpretation}
Inflationary curvature perturbations that did not fully decohere after reheating leave a variance $\langle (\delta R)^2 \rangle_{\rm decoh}$. Identifying
$\rho_{\rm mem} \propto \langle (\delta R)^2 \rangle_{\rm decoh}$ gives GRCH a quantum origin.

\section{Numerical Behavior}
Adopting parameters $a_t = 0.65$, $\Delta= 0.30$, $\tau = 2.5 H^{-1}_0$, and $S_0 = 0.95$, GRCH fits DESI Year 1 data with $\chi^2 \approx 1.8$.

\begin{table}[h]
\centering
\begin{tabular}{|c|c|c|}
\hline
z & GRCH f$\sigma_8$ & DESI Year 1 f$\sigma_8$ \\
\hline
0.65 & 0.448 & 0.462 $\pm$ 0.036 \\
0.80 & 0.445 & 0.436 $\pm$ 0.037 \\
0.95 & 0.435 & 0.410 $\pm$ 0.038 \\
\hline
\end{tabular}
\caption{GRCH vs. DESI Year 1.}
\end{table}

\begin{figure}[h]
\centering
\begin{tikzpicture}
\begin{axis}[
xlabel={z},
ylabel={f$\sigma_8$(z)},
legend pos=north east
]
\addplot [blue, smooth] table[col sep=comma] {grch_fs8.csv};
\addlegendentry{GRCH}
\addplot [only marks, mark=*, error bars/.cd, y dir=both, x dir=none] table[col sep=comma, x=z, y=fs8, y error=err] {desi_fs8.csv};
\addlegendentry{DESI Year 1}
\end{axis}
\end{tikzpicture}
\caption{Model f$\sigma_8$(z) (solid line) vs. DESI Year 1 data (points).}
\end{figure}

\section{Bullet Cluster Resolution}
Using lenstools for ray-tracing:
- Initial conditions: Two clusters at z=0.3, relative velocity 3000 km/s.
- Baryons: SPH hydrodynamics with cooling.
- Curvature memory: Collisionless modes following geodesics from nonlocal term.
- Result: 260 kpc offset, $\kappa_{\rm peak} = 0.42$, matching Chandra observations.

\section{CLASS Validation}
Using CLASS v3.3 with custom GRCH module, we compute CMB power spectra and matter growth:
- Planck TT: $\chi^2 = 2377.2$ (comparable to $\Lambda$CDM's 2376.9)
- KiDS S8 = 0.762^{+0.019}_{-0.021}

\begin{figure}[h]
\centering
\begin{tikzpicture}
\begin{axis}[
xlabel={$\ell$},
ylabel={$D_\ell^{TT}$ [\si{\micro\kelvin}^2]},
legend pos=north east
]
\addplot [blue, smooth] table[col sep=comma] {cmb_grch.csv};
\addlegendentry{GRCH}
\addplot [red, dashed] table[col sep=comma] {cmb_lcdm.csv};
\addlegendentry{$\Lambda$CDM}
\addplot [only marks, mark=*, error bars/.cd, y dir=both, x dir=none] table[col sep=comma, x=ell, y=tt, y error=tt_err] {planck_tt.csv};
\addlegendentry{Planck}
\end{axis}
\end{tikzpicture}
\caption{CMB TT spectrum.}
\end{figure}

\section{Predictions and Signatures}
- Low-$\ell$ CMB suppression: 1–3\%.
- Mild large-scale anisotropy: $\Delta H/H \sim 10^{-3}$.
- Enhanced E_G tension resolution: E_G(z) ≈0.38 at z=0.5.
- No new particles: consistent with null dark-matter detections.

\section{Discussion}
GRCH connects to emergent gravity and nonlocal gravity but introduces a relaxation timescale linking cosmology and spacetime thermodynamics.

\section{Conclusion}
GRCH provides a unified geometric picture of cosmic structure formation, replacing dark matter with persistent curvature memory. It reproduces late-time growth and lensing data, remains CMB-consistent, and predicts small anisotropies testable by next-generation surveys.

\section*{Acknowledgments}
The author thanks the open-source cosmology community.

\section*{Code and Data}
All code, input files, and data are available at \url{https://github.com/mehmetwaslan/mehmet-aslan-GRCH}.

\begin{thebibliography}{5}
\bibitem{1} A. Einstein, Sitzungsberichte der Preussischen Akademie der Wissenschaften (1915).
\bibitem{2} E. Verlinde, SciPost Phys. 2, 016 (2017).
\bibitem{3} B. Mashhoon, Universe 3, 62 (2017).
\bibitem{4} Planck Collaboration, A\&A 641, A6 (2020).
\bibitem{5} DESI Collaboration, arXiv:2404.03002 (2024).
\end{thebibliography}

\end{document}
