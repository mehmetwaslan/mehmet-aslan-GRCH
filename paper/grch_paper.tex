\documentclass[11pt,a4paper]{article}
\usepackage[margin=1in]{geometry}
\usepackage{amsmath,amssymb}
\usepackage{hyperref}
\usepackage{graphicx}
\usepackage{authblk}
\usepackage{booktabs}

\title{\textbf{The Geometric Residual Curvature Hypothesis (GRCH):}\\
A Phenomenological Curvature-Based Alternative to Dark Matter}

\author{Mehmet Aslan}
\affil{Independent Researcher, London, United Kingdom}
\date{November 2025}

\begin{document}

\maketitle

\begin{abstract}
We explore the \emph{Geometric Residual Curvature Hypothesis} (GRCH), a
phenomenological framework in which part of the effective dark-matter
behavior arises from slowly relaxing, large-scale curvature inhomogeneities
generated in the early Universe. This ``geometric memory'' component,
characterized by an effective density $\rho_{\rm mem}(a)$, an equation of
state $w(a)$, and a reduced clustering factor $S(a)$, modifies late-time
structure growth without introducing new particles.

We investigate three fully reproducible numerical tests: (1) a two-component
cluster-merger experiment illustrating baryon–memory separation,  
(2) direct numerical integration of the growth equation comparing GRCH
predictions to DESI Year~1 $f\sigma_8$ data,  
and (3) a simplified CMB TT sensitivity test showing that percent-level
clustering suppression is not excluded by Planck-like uncertainties.

The goal is not a full cosmological model, but a transparent exploration of
how residual curvature dynamics could mimic dark matter on large scales.
All code is provided in a public repository.
\end{abstract}

\section{Introduction}

The $\Lambda$CDM model provides an excellent fit to cosmological data but
requires a cold dark matter component whose microphysical nature remains
unknown. This motivates investigating whether part of the observed
``dark-matter-like'' phenomena could originate from properties of spacetime
itself.

The \emph{Geometric Residual Curvature Hypothesis} (GRCH) proposes that a
small fraction of primordial curvature fluctuations did not fully decohere
post-inflation and survive as a slowly relaxing geometric memory field. This
component behaves as a clustering fluid at early times but exhibits reduced
clustering efficiency at late times.

In this work we examine whether such a component can qualitatively reproduce
(1) merger-scale baryon–collisionless separation,  
(2) the observed growth rate from DESI,  
and (3) percent-level effects on the CMB TT spectrum.

\section{Phenomenological GRCH Framework}

We consider a flat FLRW cosmology with an additional component
$\rho_{\rm mem}(a)$ whose evolution is governed by
\begin{equation}
\frac{d\rho_{\rm mem}}{d\ln a}
=
-3(1+w(a))\rho_{\rm mem}
-\frac{\rho_{\rm mem}}{H\tau},
\label{eq:mem_bg}
\end{equation}
where $w(a)$ is a phenomenological equation of state and $\tau$ a relaxation
timescale.

A smooth transition between early-time matter-like behavior and late-time
mild vacuum-like behavior is implemented via
\begin{equation}
w(a)
=
w_{\rm late}
+\frac{1}{2} (w_{\rm early}-w_{\rm late})
\left[1 - \tanh\!\left(\frac{\ln a - \ln a_t}{\Delta}\right)\right],
\end{equation}
with $w_{\rm early}\simeq 0$, $w_{\rm late}\simeq -(1-S_0)$.

Reduced clustering efficiency is encoded in
\begin{equation}
S(a)
=
S_0
+\frac{1-S_0}{2}
\left[1-
\tanh\!\left(\frac{\ln a-\ln a_t}{0.5}\right)
\right].
\end{equation}

The Friedmann equation is
\begin{equation}
H^2(a)=H_0^2
\left[
\Omega_b a^{-3}
+\Omega_r a^{-4}
+\Omega_\Lambda
+ \frac{\rho_{\rm mem}(a)}{\rho_{\rm crit,0}}
\right].
\end{equation}

The linear growth rate $f=d\ln D/d\ln a$ obeys
\begin{equation}
\frac{df}{d\ln a}
+f^2
+\left[2+\frac{d\ln H}{d\ln a}\right]f
=
\frac{3}{2}\Omega_{\rm cl}(a),
\end{equation}
with effective clustering density
\begin{equation}
\Omega_{\rm cl}(a)
=
\Omega_b a^{-3}
+
\frac{S(a)\rho_{\rm mem}(a)/\rho_{\rm crit,0}}
{H^2(a)/H_0^2}.
\end{equation}

All parameter values used in our numerical tests are purely
phenomenological and documented in the accompanying script
\texttt{derive\_parameters.py}.

\section{Cluster–Scale Dynamics}

To assess whether GRCH can produce baryon–collisionless separation on merger
scales, we perform a minimal two-component experiment using
\texttt{run\_bullet.py}. Each $10^{15}M_\odot$ cluster consists of:

\begin{itemize}
\item a collisional baryonic component subject to an effective drag,  
\item a collisionless geometric-memory component evolving under gravity.
\end{itemize}

With an initial relative velocity of $3000\,{\rm km\,s^{-1}}$ and softening
length $30\,{\rm kpc}$, the baryonic and memory components decouple after
core passage, producing

\[
\Delta x \simeq 260\,{\rm kpc}, \quad
\kappa_{\rm peak} \simeq 0.41.
\]

These results demonstrate that a reduced-clustering, collisionless memory
component can generate Bullet-like offsets under realistic merger
kinematics.

\section{Growth Rate and DESI Comparison}

We integrate the modified growth equation using \texttt{run\_grch\_growth.py}.
For parameters 
$S_0=0.95$,
$a_t=0.65$,
$\tau=2.5H_0^{-1}$,
$\sigma_8(0)=0.81$,
we obtain:

\begin{center}
\begin{tabular}{c|c|c}
\toprule
$z$ & GRCH $f\sigma_8$ & DESI Y1 \\
\midrule
0.65 & 0.463 & $0.462 \pm 0.036$ \\
0.80 & 0.451 & $0.436 \pm 0.037$ \\
0.95 & 0.436 & $0.410 \pm 0.038$ \\
\bottomrule
\end{tabular}
\end{center}

The three points yield a simple $\chi^2\simeq 0.65$.  
The deviation from $\Lambda$CDM is controlled primarily by the late-time
value of $S(a)$, indicating that GRCH can reproduce observed growth
suppression without modifying the early Universe.

\section{CMB TT Sensitivity}

We assess whether the percent-level reduction in late-time clustering
implied by GRCH is consistent with CMB TT data. Instead of a full Boltzmann
treatment, we use a simplified TT template based on $\Lambda$CDM-like
$D_\ell^{TT}$ values and apply a uniform $1.5\%$ suppression representing the
change in clustering amplitude between $z=2$ and today.  

Comparison with approximate Planck-like points shows that the resulting
difference is smaller than observational uncertainties at all multipoles,
including the low-$\ell$ ISW-dominated region. A full Boltzmann
implementation of GRCH is left for future work.

\section{Discussion}

Our results suggest that a slowly relaxing geometric memory component with
reduced late-time clustering can qualitatively reproduce:

\begin{itemize}
\item separation between collisional and collisionless matter in high-speed
cluster mergers,
\item the observed suppression of the linear growth rate from DESI,
\item and percent-level CMB TT sensitivity.
\end{itemize}

GRCH is not yet a full cosmological alternative; rather, it offers a
minimal, phenomenologically motivated framework showing how curved-spacetime
memory could mimic aspects of dark matter.

A natural next step is incorporating GRCH into a Boltzmann code (CLASS or
CAMB), enabling predictions for $C_\ell^{TT}$, $C_\ell^{\phi\phi}$, and
$S_8$.

\section{Conclusion}

The Geometric Residual Curvature Hypothesis provides a coherent phenomenological
framework for exploring curvature-based alternatives to dark matter. With
transparent numerical demonstrations and a clear theoretical motivation,
GRCH offers a promising avenue for further development in cosmological
modeling.

\section*{Code Availability}
All numerical scripts and parameter files are publicly available at:\\
\url{https://github.com/mehmetwaslan/mehmet-aslan-GRCH}.

\end{document}
