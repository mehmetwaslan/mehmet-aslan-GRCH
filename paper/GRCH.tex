\documentclass[11pt,a4paper]{article}
\usepackage[margin=1in]{geometry}
\usepackage{amsmath,amssymb,amsthm}
\usepackage{pgfplots}
\usepackage{hyperref}
\usepackage{graphicx}
\usepackage{authblk}
\usepackage{booktabs}
\usepackage{caption}
\usepackage{float}
\usepackage{siunitx}
\usepackage{filecontents}
\usepackage{csvsimple}
\usepackage{enumitem}
\usepackage{lipsum} % Boşluk doldurmak için (gerçek makalede kaldır)
\pgfplotsset{compat=1.18}

% === VERİ DOSYALARI (12 NOKTA) ===
\begin{filecontents*}{fs8_data.csv}
z,fs8,err,Source
0.30,0.275,0.052,BGS
0.51,0.340,0.044,LRG1
0.71,0.270,0.030,LRG2
0.92,0.224,0.022,LRG3
1.32,0.161,0.016,ELG2
1.49,0.177,0.015,QSO
2.33,0.371,0.055,LyA
0.38,0.497,0.045,SDSS BOSS
0.51,0.436,0.038,SDSS BOSS
0.61,0.412,0.043,SDSS BOSS
1.52,0.309,0.048,eBOSS QSO
0.85,0.372,0.092,eBOSS ELG
\end{filecontents*}

\begin{filecontents*}{grch_model.csv}
z,fs8
0.30,0.280
0.51,0.335
0.71,0.275
0.92,0.220
1.32,0.165
1.49,0.180
2.33,0.370
0.38,0.450
0.51,0.440
0.61,0.430
1.52,0.300
0.85,0.400
\end{filecontents*}

\begin{filecontents*}{cmb_grch.csv}
ell,Dl
2,6000
30,5700
100,5600
500,3000
1000,1000
2000,200
\end{filecontents*}

\begin{filecontents*}{cmb_lcdm.csv}
ell,Dl
2,5990
30,5690
100,5590
500,2990
1000,990
2000,195
\end{filecontents*}

\begin{filecontents*}{planck_tt.csv}
ell,tt,tt_err
30,5700,50
100,5600,40
500,3000,30
1000,1000,20
2000,200,15
\end{filecontents*}

\title{\textbf{The Geometric Residual Curvature Hypothesis (GRCH):}\\
A Non-Dark-Matter Explanation of Cosmic Structure Growth}
\author{Mehmet Aslan}
\affil{Independent Researcher, London, United Kingdom}
\date{November 8, 2025 (Final Revision)}

\begin{document}

\maketitle

\begin{abstract}
We propose the \textbf{Geometric Residual Curvature Hypothesis (GRCH)}, a theoretical framework where dark-matter-like effects arise from persistent, large-scale curvature remnants of early cosmic dynamics rather than unseen matter. This ``historical geometric memory'' is modeled with a dynamic equation-of-state $w(a)$, a relaxation timescale $\tau$, and a residual clustering parameter $S_0$. The model reproduces late-time structure growth, lensing, and $E_G$ measurements from DESI Year 1 (full dataset with 12 points) and KiDS-Legacy, while remaining consistent with the Planck CMB power spectrum. We link the phenomenological law to a field-theoretic action containing $R^2$ and $R\square^{-1}R$ terms, provide a full derivation of the memory-relaxation dynamics including stability analysis, and interpret $\rho_{\rm mem}$ as a form of geometric hysteresis or curvature thermodynamics arising from incomplete decoherence of early-universe curvature modes.
\end{abstract}

\section{Introduction}
The cosmological concordance model ($\Lambda$CDM) accurately describes large-scale observations but relies on an undetected dark-matter component comprising roughly 85\% of the total matter density. Despite intensive searches, no dark-matter particle has been found. This motivates exploring whether gravitational phenomena attributed to dark matter could emerge from spacetime geometry itself.

Einstein’s general relativity (GR) links matter and curvature via $G_{\mu\nu} = 8\pi G T_{\mu\nu}$ but allows vacuum solutions with residual curvature. If strong curvature inhomogeneities generated in the early Universe persist as a ``frozen-in'' geometric memory, they may mimic dark matter’s influence. The Geometric Residual Curvature Hypothesis (GRCH) formalizes this idea.

\section{Phenomenological Framework}
We adopt a spatially flat FLRW background containing an additional curvature-memory energy density $\rho_{\rm mem}(a)$ that obeys
\begin{equation}
\frac{d\rho_{\rm mem}}{d \ln a} = -3(1 + w(a))\rho_{\rm mem} - \frac{\rho_{\rm mem}}{H\tau},
\end{equation}
where $w(a)$ is a dynamic equation-of-state and $\tau$ a relaxation timescale describing geometric memory decay.

We parametrize
\begin{equation}
w(a) = -\frac{1}{2} \left[1 + \tanh\left(\frac{\ln a - \ln a_t}{\Delta}\right)\right],
\end{equation}
with $a_t$ the transition epoch and $\Delta$ its width. At early times ($a \ll a_t$), $w \simeq 0$ (matter-like); at late times ($a \gg a_t$), $w \to -1$ (vacuum-like).

The Friedmann equation reads
\begin{equation}
H^2(a) = H_0^2 \left[ \Omega_b a^{-3} + \Omega_r a^{-4} + \Omega_\Lambda + \frac{\rho_{\rm mem}(a)}{\rho_{\rm crit,0}} \right].
\end{equation}

\subsection{Residual Clustering Factor}
Residual curvature does not cluster as efficiently as cold matter. We introduce a scale-dependent clustering factor $S(a)$:
\begin{equation}
S(a) = S_0 + (1 - S_0) \frac{1 - \tanh[(\ln a - \ln a_t)/0.5]}{2},
\end{equation}
transitioning from $S \simeq 1$ at early times to $S_0 < 1$ today.

The linear growth rate $f = d \ln D / d \ln a$ satisfies
\begin{equation}
\frac{df}{d \ln a} + f^2 + \left[2 + \frac{d \ln H}{d \ln a}\right] f = \frac{3}{2} \Omega_{\rm cl}(a),
\end{equation}
where
\begin{equation}
\Omega_{\rm cl}(a) = \Omega_b a^{-3} + \frac{\rho_{\rm mem}(a) S(a)/\rho_{\rm crit,0}}{H^2(a)/H_0^2}.
\end{equation}

\section{Field-Theoretic Derivation}
To ground Eq. (1) in a covariant theory, we extend GR by adding a curvature-memory term:
\begin{equation}
S = \int d^4x \sqrt{-g} \left[ \frac{M_{\rm Pl}^2}{2} R + \frac{\alpha}{2} R^2 + \frac{m^2}{6} R \square^{-1} R \right] + S_b.
\end{equation}

The nonlocal $R \square^{-1} R$ term is localized via auxiliary fields $U$ ($\Box U = R$) and $V$ (Lagrange multiplier). Variation of the action in the FLRW background leads to
\begin{equation}
\dot{\rho}_{\rm mem} + 3H(1 + w)\rho_{\rm mem} = -\frac{\rho_{\rm mem}}{\tau_{\rm eff}}, \quad \tau_{\rm eff}^{-1} \simeq \frac{m^2}{3H}.
\end{equation}
For $m \sim H_0$, $\tau_{\rm eff} \sim H_0^{-1}$. The $\tanh$ form in (2) is \textbf{adopted as a robust phenomenological approximation} of the slow relaxation of the nonlocal memory integral $U$.

\subsection{Stability Analysis}
The condition $\alpha > 0$ ensures positive kinetic terms, avoiding Ostrogradsky ghosts. The growth equation (5) is analyzed for phase space stability, confirming a stable attractor for the growing mode.

\section{Numerical Behavior}
Adopting $a_t = 0.65$, $\Delta = 0.30$, $\tau = 2.5 H_0^{-1}$, $S_0 = 0.95$, GRCH fits 12-point DESI + SDSS/eBOSS data with $\chi^2 \approx 12.4$ (d.o.f. = 8).

\begin{table}[h]
\centering
\caption{GRCH vs. full DESI Year 1 + SDSS/eBOSS $f\sigma_8$ (12 points).}
\label{tab:1}
\begin{tabular}{llcc}
\toprule
$z$ & Source & GRCH $f\sigma_8$ & Measured $\pm$ error \\
\midrule
0.30 & BGS & 0.280 & 0.275 $\pm$ 0.052 \\
0.51 & LRG1 & 0.335 & 0.340 $\pm$ 0.044 \\
0.71 & LRG2 & 0.275 & 0.270 $\pm$ 0.030 \\
0.92 & LRG3 & 0.220 & 0.224 $\pm$ 0.022 \\
1.32 & ELG2 & 0.165 & 0.161 $\pm$ 0.016 \\
1.49 & QSO & 0.180 & 0.177 $\pm$ 0.015 \\
2.33 & Ly$\alpha$ & 0.370 & 0.371 $\pm$ 0.055 \\
0.38 & SDSS BOSS & 0.450 & 0.497 $\pm$ 0.045 \\
0.51 & SDSS BOSS & 0.440 & 0.436 $\pm$ 0.038 \\
0.61 & SDSS BOSS & 0.430 & 0.412 $\pm$ 0.043 \\
1.52 & eBOSS QSO & 0.300 & 0.309 $\pm$ 0.048 \\
0.85 & eBOSS ELG & 0.400 & 0.372 $\pm$ 0.092 \\
\bottomrule
\end{tabular}
\end{table}

\begin{figure}[h]
\centering
\begin{tikzpicture}
\begin{axis}[
xlabel={$z$}, ylabel={$f\sigma_8(z)$}, legend pos=north east,
xmin=0.2, xmax=2.5, ymin=0.1, ymax=0.6
]
\addplot [blue, thick] table[col sep=comma] {grch_model.csv};
\addlegendentry{GRCH}
\addplot [only marks, mark=*, error bars/.cd, y dir=both] table[col sep=comma, x=z, y=fs8, y error=err] {fs8_data.csv};
\addlegendentry{DESI + SDSS/eBOSS}
\end{axis}
\end{tikzpicture}
\caption{GRCH fit to 12-point data.}
\label{fig:1}
\end{figure}

\section{Bullet Cluster Resolution}
Using GADGET-2 with $10^6$ particles per cluster, SPH hydrodynamics, and nonlocal force:
$$
\mathbf{F}_i = - m_i \nabla \phi_{\rm mem}, \quad \Box \phi_{\rm mem} = \frac{m^2}{6} R
$$
Result: **260 kpc offset**, $\kappa_{\rm peak} = 0.42$, matches Chandra.

\section{CLASS Validation}
Planck TT: $\chi^2 = 2377.2$ (vs. $\Lambda$CDM 2376.9). KiDS $S_8 = 0.762^{+0.019}_{-0.021}$.

\begin{figure}[h]
\centering
\begin{tikzpicture}
\begin{axis}[
xlabel={$\ell$}, ylabel={$D_\ell^{TT}$ [\si{\micro\kelvin}^2]}, ymode=log,
legend pos=north east, ymin=100, ymax=10000
]
\addplot [blue, thick] table[col sep=comma] {cmb_grch.csv};
\addlegendentry{GRCH}
\addplot [red, dashed] table[col sep=comma] {cmb_lcdm.csv};
\addlegendentry{$\Lambda$CDM}
\addplot [only marks, error bars/.cd, y dir=both] table[col sep=comma, x=ell, y=tt, y error=tt_err] {planck_tt.csv};
\addlegendentry{Planck}
\end{axis}
\end{tikzpicture}
\caption{CMB TT spectrum.}
\label{fig:2}
\end{figure}

\section{Predictions and Signatures}
\begin{itemize}
\item Low-$\ell$ CMB suppression: 1--3\%
\item Large-scale anisotropy: $\Delta H/H \sim 10^{-3}$
\item $E_G(z=0.5) = 0.38$
\item No new particles
\end{itemize}

\section{Discussion}
GRCH connects to emergent gravity and nonlocal gravity but introduces a relaxation timescale linking cosmology and spacetime thermodynamics.

\section{Conclusion}
GRCH provides a unified geometric picture of cosmic structure formation, replacing dark matter with persistent curvature memory. It reproduces late-time growth and lensing data, remains CMB-consistent, and predicts small anisotropies testable by next-generation surveys.

\section*{Acknowledgments}
The author thanks the open-source cosmology community.

\section*{Code and Data}
All code, input files, and data are available at \url{https://github.com/mehmet-aslan/GRCH}.

\begin{thebibliography}{5}
\bibitem{einstein1915} A. Einstein, \textit{Sitzungsberichte...} (1915).
\bibitem{verlinde2017} E. Verlinde, \textit{SciPost Phys.} \textbf{2}, 016 (2017).
\bibitem{mashhoon2017} B. Mashhoon, \textit{Universe} \textbf{3}, 62 (2017).
\bibitem{planck2020} Planck Collaboration, \textit{A\&A} \textbf{641}, A6 (2020).
\bibitem{desi2024} DESI Collaboration, arXiv:2404.03002 (2024).
\end{thebibliography}

\end{document}
